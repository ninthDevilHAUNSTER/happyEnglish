\documentclass[a4paper, 12pt]{article}
\usepackage[UTF8]{ctex}
\usepackage{multicol}
\usepackage{geometry}
\geometry{a4paper,left=2cm,right=2cm,top=1cm,bottom=1cm}

\begin{document}
    \noindent

    \title{每\ 日\ 单\ 词\ }
    \author{烧包}
    \maketitle

\begin{flushleft}
一. 单词部分
\end{flushleft}

\begin{multicols}{2}
\begin{flushleft}
1.\ 爱因斯坦(汀) \ \ \ \ \underline{\hspace{3cm}}
\end{flushleft}

\begin{flushleft}
2.\ 多元文化(马赛克) \ \ \ \ \underline{\hspace{3cm}}
\end{flushleft}
\end{multicols}

\begin{multicols}{2}
\begin{flushleft}
3.\ 偶像崇拜 \ \ \ \ \underline{\hspace{3cm}}
\end{flushleft}

\begin{flushleft}
4.\ 莎士比亚(摇茅e) \ \ \ \ \underline{\hspace{3cm}}
\end{flushleft}
\end{multicols}

\begin{multicols}{2}
\begin{flushleft}
5.\ 启蒙主义 \ \ \ \ \underline{\hspace{3cm}}
\end{flushleft}

\begin{flushleft}
6.\ 解读经典 \ \ \ \ \underline{\hspace{3cm}}
\end{flushleft}
\end{multicols}

\begin{multicols}{2}
\begin{flushleft}
7.\ 上网成瘾 \ \ \ \ \underline{\hspace{3cm}}
\end{flushleft}

\begin{flushleft}
8.\ 解构主义 \ \ \ \ \underline{\hspace{3cm}}
\end{flushleft}
\end{multicols}

\begin{multicols}{2}
\begin{flushleft}
9.\ 人本主义 \ \ \ \ \underline{\hspace{3cm}}
\end{flushleft}

\begin{flushleft}
10.\ 文化融合 \ \ \ \ \underline{\hspace{3cm}}
\end{flushleft}
\end{multicols}

\begin{multicols}{2}
\begin{flushleft}
11.\ 积极醋精文化发展 \ \ \ \ \underline{\hspace{3cm}}
\end{flushleft}

\begin{flushleft}
12.\ 主流文化 \ \ \ \ \underline{\hspace{3cm}}
\end{flushleft}
\end{multicols}

\begin{multicols}{2}
\begin{flushleft}
13.\ 手机入网 \ \ \ \ \underline{\hspace{3cm}}
\end{flushleft}

\begin{flushleft}
14.\ 网络监管 \ \ \ \ \underline{\hspace{3cm}}
\end{flushleft}
\end{multicols}

\begin{multicols}{2}
\begin{flushleft}
15.\ 捐赠;定金;签署 \ \ \ \ \underline{\hspace{3cm}}
\end{flushleft}

\begin{flushleft}
16.\ 后现代主义 \ \ \ \ \underline{\hspace{3cm}}
\end{flushleft}
\end{multicols}

\Large{二. 句子部分}

\begin{flushleft}
1.\ 三纲五常(儒道佛;仁义礼智信)

\underline{\hspace{16cm}}
\end{flushleft}

\begin{flushleft}
2.\ 辉煌璀璨的文化遗产

\underline{\hspace{16cm}}
\end{flushleft}

\begin{flushleft}
3.\ 取其精华去其糟粕

\underline{\hspace{16cm}}
\end{flushleft}

\end{document}
